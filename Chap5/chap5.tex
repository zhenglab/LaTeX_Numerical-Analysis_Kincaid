\documentclass[notheorems,mathserif,table,compress]{beamer}  %dvipdfm选项是关键,否则编译统统通不过
%%------------------------常用宏包------------------------
%%注意, beamer 会默认使用下列宏包: amsthm, graphicx, hyperref, color, xcolor, 等等
\usepackage{fontspec,xunicode,xltxtra}
\usepackage{amsfonts,amssymb}  % for XeTeX
%%------------------------ThemeColorFont------------------------
%% Presentation Themes
% \usetheme[<options>]{<name list>}
\usetheme{Madrid}
%% Inner Themes
% \useinnertheme[<options>]{<name>}
%% Outer Themes
% \useoutertheme[<options>]{<name>}
\useoutertheme{miniframes} 
%% Color Themes 
% \usecolortheme[<options>]{<name list>}
%% Font Themes
% \usefonttheme[<options>]{<name>}
\setbeamertemplate{background canvas}[vertical shading][bottom=white,top=structure.fg!7] %%背景色, 上25%的蓝, 过渡到下白.
\setbeamertemplate{theorems}[numbered]
\setbeamertemplate{navigation symbols}{}   %% 去掉页面下方默认的导航条.
\usepackage{zhfontcfg}
\usepackage{iplouclistings}
%\setsansfont[Mapping=tex-text]{文泉驿正黑}  %% 需要fontspec宏包
     %如果装了Adobe Acrobat,可在font.conf中配置Adobe字体的路径以使用其中文字体
     %也可直接使用系统中的中文字体如SimSun,SimHei,微软雅黑 等
     %原来beamer用的字体是sans family;注意Mapping的大小写,不能写错
     %设置字体时也可以直接用字体名,以下三种方式等同:
     %\setromanfont[BoldFont={黑体}]{宋体}
     %\setromanfont[BoldFont={SimHei}]{SimSun}
     %\setromanfont[BoldFont={"[simhei.ttf]"}]{"[simsun.ttc]"}
%%------------------------MISC------------------------
\graphicspath{{figures/}}         %% 图片路径. 本文的图片都放在这个文件夹里了.
%%------------------------正文------------------------
\begin{document}
\XeTeXlinebreaklocale "zh"         % 表示用中文的断行
\XeTeXlinebreakskip = 0pt plus 1pt % 多一点调整的空间
%%----------------------------------------------------------
%% This is only inserted into the PDF information catalog. Can be left
%% out.
%%%
%% Delete this, if you do not want the table of contents to pop up at
%% the beginning of each subsection:
\AtBeginSection[]{                              % 在每个Section前都会加入的Frame
  \frame<handout:0>{
    \frametitle{内容提要}\small
    \tableofcontents[current,currentsubsection]
  }
}
\AtBeginSubsection[]                            % 在每个子段落之前
{
  \frame<handout:0>                             % handout:0 表示只在手稿中出现
  {
    \frametitle{下一节内容}\small
    \tableofcontents[current,currentsubsection] % 显示在目录中加亮的当前章节
  }
}
%%----------------------------------------------------------
\title[Numerical Analysis]{Numerical Analysis}
\subtitle{Mathematics of Scientific Computing}
\author[sun]{Name~~~~~\textcolor{olive}{Sun Xiaoqing}\\
      \quad 幻灯片制作~~\textcolor{olive}{孙晓庆}}
   
\institute[中国海洋大学]{\small\textcolor{violet}{中国海洋大学~~信息科学与工程学院}}
\date{2013~年~11~月~21~日}
%\titlegraphic{\vspace{-6em}\includegraphics[height=7cm]{ouc}\vspace{-6em}}
\frame{ \titlepage }
%%----------------------------------------------------------
\section*{目录}
\frame{\frametitle{目录}\tableofcontents}
%%----------------------------------------------------------
\section{Matrix Eigenvalue Problem by Power Methods}
\subsection{Power Method}%如果你想书签不出现问题,请不要用\XeTeX
                                 %这类复杂的指令,直接写XeTeX吧
\begin{frame}
To compute the dominant eigenvalue of $A$,\\
A has the following two properties:
\begin{itemize}
\item There is a single eigenvalue of maximum modulus.
\item There is a linearly independent set of $n$ eigenvectors.
\end{itemize}
\end{frame}

\begin{frame}
Two assumpting:
\begin{itemize}
\item The eigenvalues $\lambda_{1}$, $\lambda_{2}$,..., $\lambda_{n}$ can be labed $|\lambda_{1}|>|\lambda_{2}|\geq ... \geq|\lambda_{n}|$.
\item There is a basic ${u^{(1)}, u^{(2)},..., u^{(n)}}$, for that $Au^{(j)}=\lambda_{j}u^{(j)}$
\end{itemize}
\end{frame}


\begin{frame}
$x^{(0)}=a_{1}u^{(1)}+a_{2}u^{(2)}+...+a_{n}u^{(n)}$\qquad $(a_{1}\neq0)$\\
$x^{(1)}=Ax^{(0)}$ \qquad $x^{(2)}=Ax^{(1)}$ ... \qquad $x^{(k)}=Ax^{(k-1)}$\\
$x^{(k)}=A^{k}x^{(0)}$\\
$x^{(0)}=u^{(1)}+u^{(2)}+...+u^{(n)}$\\
$x^{(k)}=A^{k}u^{(1)}+A^{k}u^{(2)}+...+A^{k}u^{(n)}$\\
$x^{(k)}=\lambda_{1}^{k}u^{(1)}+\lambda_{2}^{k}u^{(2)}+...+\lambda_{k}^{k}u^{(n)}$\\
$x^{(k)}=\lambda_{1}^{k}[u^{(1)}+(\lambda_{2}/\lambda_{1})^{k}u^{(2)}+...+(\lambda_{n}/\lambda_{1})^{k}u^{(n)}]$\\
Since $|\lambda_{1}|>|\lambda_{j}|$, so $k\rightarrow \infty$, $(\lambda_{j}/\lambda_{1})^{k}\rightarrow 0$\\
$x^{(k)}=\lambda_{1}^{k}[u^{(1)}+\varepsilon^{(k)}]$\\
$\varphi (x^{(k)})=\lambda_{1}^{k}[\varphi(u^{(1)})+\varphi(\varepsilon^{(k)})]$\\
When$k\rightarrow \infty$, $r_{k}=\frac{\varphi (x^{(k+1)})}{\varphi (x^{(k)})}=\lambda_{1}\frac{[\varphi(u^{(1)})+\varphi(\varepsilon^{(k+1)})]}{[\varphi(u^{(1)})+\varphi(\varepsilon^{(k)})]}\rightarrow \lambda_{1}$
\end{frame}
       
% \XeTeXpicfile "./logo.jpg" xscaled 100 yscaled 100 %插图也没有问
\begin{frame}
Algorithm of Power Method\\
input $A$, $x$, $M$\\
for $k=1$ to $M$ do\\
$y\leftarrow Ax$\\
$r\leftarrow \varphi(y)/\varphi(x)$\\
$x\leftarrow y$\\
output $k$, $x$, $r$\\
end do\\
\end{frame}

\begin{frame}
\lstinputlisting{./code/Powermethod/powerMethod.m}
\lstinputlisting{./code/Powermethod/linearFun.m}
\end{frame}

\subsection{Inverse Power Method}
\begin{frame}
The inverse power method comput the smallest engenvalue of $A$.\\
Eigenvalues of $A$ can be arranged as follows:\\
$|\lambda_{1}|>|\lambda_{2}|\geq ... \geq|\lambda_{n}|>0$.\\
Eigenvalues of $A^{-1}$ can be arranged as follows:\\
$|\lambda_{n}^{-1}|>|\lambda_{n-1}^{-1}|\geq ... \geq|\lambda_{1}^{-1}|>0$.\\
$x^{(k+1)}=A^{-1}x^{(k)}$
\end{frame}

\begin{frame}
\lstinputlisting{./code/inversepower.m}
\lstinputlisting{./code/Powermethod/linearFun.m}
\end{frame}



\subsection{Summary}
\begin{frame}
\frametitle{Summary}
\begin{tabular}{lllrrrcc}
Method          & Equation & Computes\\
power           &    $x^{(k+1)}=Ax^{(k)}$ &lagest eigenvalue\\
inverse power  & $Ax^{(k+1)}=x^{(k)}$ & smallest eigenvalue\\
shifted power  & $x^{(k+1)}=(A-uI)x^{(k)}$ &  eigenvalue farthest from $u$\\
shifted inverse power  & $(A-uI)x^{(k+1)}=x^{(k)}$ &  eigenvalue closest to $u$\\
\end{tabular}
\end{frame}





\begin{frame}
The code of shifted power
\lstinputlisting{./code/shiftpower.m}
\lstinputlisting{./code/Powermethod/linearFun.m}
\end{frame}

\begin{frame}
The code of shifted inverse power
\lstinputlisting{./code/invshiftpower.m}
\lstinputlisting{./code/Powermethod/linearFun.m}
\end{frame}



\section{QR-Factorization}

\begin{frame}
\frametitle{QR-Factorization}
$A=QR$\\
$A$ is an $m\times n$ matrix, $Q$ is an $m\times m$ unitary matrix and $R$ is an $m\times n$ upper triangular martrix. 
\end{frame}


\begin{frame}

$A_{1}=A$\\
$A_{1}=Q_{1}R_{1}$\\
$A_{2}=R_{1}Q_{1}$, $A_{2}\sim A_{1}=A$\\
$A_{2}=Q_{2}R_{2}$\\
$A_{3}=R_{2}Q_{2}$, $A_{3}\sim A_{2}\sim A_{1}=A$\\
we have iterative process as follows:\\
$A_{k}=Q_{k}R_{k}$  $(A_{1}=A)$\\
$A_{k+1}=R_{k}Q_{k}$,  $(k=1,2\cdots \cdots )$\\
$A_{k+1}\sim A$\\
$A_{k}$ converges to the upper triangular matrix whose diagonal elements are $\lambda_{1}, \lambda_{2}, \cdots \cdots \lambda_{n}$.

\end{frame}




\end{document}
