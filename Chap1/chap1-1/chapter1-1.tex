\documentclass[notheorems,mathserif,table,compress]{beamer}  %dvipdfm选项是关键,否则编译统统通不过
%%------------------------常用宏包------------------------
%%注意, beamer 会默认使用下列宏包: amsthm, graphicx, hyperref, color, xcolor, 等等
\usepackage{fontspec,xunicode,xltxtra}  % for XeTeX
\usepackage{verbatim}
\usepackage{mathabx}
%%------------------------ThemeColorFont------------------------
%% Presentation Themes
% \usetheme[<options>]{<name list>}
\usetheme{Madrid}
%% Inner Themes双精度计算
% \useinnertheme[<options>]{<name>}
%% Outer Themes
% \useoutertheme[<options>]{<name>}
\useoutertheme{miniframes} 
%% Color Themes 
% \usecolortheme[<options>]{<name list>}
%% Font Themes
\usefonttheme{serif}
\setbeamertemplate{background canvas}[vertical shading][bottom=white,top=structure.fg!7] %%背景色, 上25%的蓝, 过渡到下白.
\setbeamertemplate{theorems}[numbered]
\setbeamertemplate{navigation symbols}{}   %% 去掉页面下方默认的导航条.
\usepackage{zhfontcfg}
%\setsansfont[Mapping=tex-text]{文泉驿正黑}  %% 需要fontspec宏包
     %如果装了Adobe Acrobat,可在font.conf中配置Adobe字体的路径以使用其中文字体
     %也可直接使用系统中的中文字体如SimSun,SimHei,微软雅黑 等
     %原来beamer用的字体是sans family;注意Mapping的大小写,不能写错
     %设置字体时也可以直接用字体名,以下三种方式等同:
     %\setromanfont[BoldFont={黑体}]{宋体}
     %\setromanfont[BoldFont={SimHei}]{SimSun}
     %\setromanfont[BoldFont={"[simhei.ttf]"}]{"[simsun.ttc]"}
%%------------------------MISC------------------------
\graphicspath{{figures/}}         %% 图片路径. 本文的图片都放在这个文件夹里了.
%%------------------------正文------------------------
\begin{document}
\XeTeXlinebreaklocale "zh"         % 表示用中文的断行
\XeTeXlinebreakskip = 0pt plus 1pt % 多一点调整的空间
%%----------------------------------------------------------
%% This is only inserted into the PDF information catalog. Can be left
%% out.
%%%
%% Delete this, if you do not want the table of contents to pop up at
%% the beginning of each subsection:
\AtBeginSection[]{                              % 在每个Section前都会加入的Frame
  \frame<handout:0>{
    \frametitle{Contents}\small
    \tableofcontents[current,currentsubsection]
  }
}

\AtBeginSubsection[]                            % 在每个子段落之前
{
  \frame<handout:0>                             % handout:0 表示只在手稿中出现
  {
    \frametitle{Contents}\small
    \tableofcontents[current,currentsubsection] % 显示在目录中加亮的当前章节
  }
}

%%----------------------------------------------------------
\title[Numerical Analysis]{Numerical Analysis}
\subtitle{Mathematics of Scientific Computing}
\author[qiu]{主讲人~~~~~\textcolor{olive}{邱欣欣}\\
    \quad 幻灯片制作~~\textcolor{olive}{邱欣欣}}
\institute[中国海洋大学]{\small\textcolor{violet}{中国海洋大学~~信息科学与工程学院}}
\date{2013~年~9~月~6~日}
%\titlegraphic{\vspace{-6em}\includegraphics[height=7cm]{ouc}\vspace{-6em}}
\frame{ \titlepage }
%%----------------------------------------------------------
\section*{Contents}
\frame{\frametitle{Contents}\tableofcontents}



%%----------------------------------------------------------
\section{Basic Concepts and Taylor's Theorem}

\subsection{Limit, Continuity, and Derivative}

%如果你想书签不出现问题,请不要用\XeTeX
                                 %这类复杂的指令,直接写XeTeX吧

\begin{frame}
  \frametitle{Limit, Continuity, and Derivative}
  If \emph{f} is a real-valued function of a real variable, then the limit of the function \emph{f} at \emph{c} (if it exists)is defined as the following:
\begin{displaymath}
\lim_{x \to c}f(x)=L
\end{displaymath}
means that to each positive $\varepsilon$ there corresponds a positive $\delta$ such that the distance between $f(x)$ and $L$ is less than $\varepsilon$ whenever the distance between $x$ and $c$ is less than $\delta$ ; that is $|f(x)−L|<\varepsilon$ whenever $0<|x−c|<\delta$.
\\If there is no number $L$ with this property,the limit of $f$ at $c$ does not exist.
\end{frame}

\begin{frame}
  \frametitle{Limit, Continuity, and Derivative}
 If $f$ is defined only on a specified subset $X$ of
the real line, the definition of limit is modified
so that $|f(x)−L| < \varepsilon$ whenever $x \in X$ and $0<|x−c|<\delta$.
\newline

The function $f$ is said to be continuous at $c$ if
\begin{displaymath}
\lim_{x \to c}f(x)=f(c)
\end{displaymath}
\end{frame}

\begin{frame}
  \frametitle{Limit, Continuity, and Derivative}
\begin{description}
\item[THEOREM 1] \textsf{Intermediate-Value Theorem for Continuous Functions}
\end{description}
On an interval [a,b], a continuous function assumes all values between $f(a)$ and $f(b)$. 
\newline

The derivative of $f$ at $c$ (if it exists)
is defined by the equation
\begin{displaymath}
f '(c)=\lim_{x \to c}\frac{f(x)-f(c)}{x-c}
\end{displaymath}
If $f$ is a function for which $f '(c)$ exists , we say that $f$ is differentiable at $c$, then $f$ must be
continuous at $c$, $f '(x)$ exists and 
$\lim_{x \to c}f(x)=f(c)$.
\end{frame}

\subsection{Taylor's Theorem}
\begin{frame}
  \frametitle{Taylor's Theorem}
\begin{description}
\item[THEOREM 2] \textsf{Taylor's Theorem with Lagrange Remainder}
\end{description}
If $f \in C^ n [a,b] $and if $f^{(n+1)}$ exists on the open interval (a,b), then for any points $c$  and $x$ in the closed interval [a,b], 
\begin{displaymath}
f(x)=\sum_{k=0}^n \frac{1}{k!}f^{(k)}(c)(x-c)^k +E_n(x)
\end{displaymath}
where , for some point $\xi$ between $c$ and $x$ , the error term is
\begin{displaymath}
E_n(x)=\frac{1}{(n+1)!}f^{(n+1)}(\xi)(x-c)^{(n+1)}
\end{displaymath}
Here ``$\xi$ between $c$ and $x$''means that either $c <\xi < x$ or $x < \xi <c$
depending on the particular values of $x$ and $c$ involved.
\end{frame}

\begin{frame}
  \frametitle{Taylor's Theorem}
An important special case arises when $c=0$. The equation becomes the \mbox{\boldmath {$Maclaurin$} {$series$}} for $f(x)$:
\begin{displaymath}
f(x)=\sum_{k=0}^n \frac{1}{k!}f^{(k)}(0)x^k +E_n(x)
\end{displaymath}
where 
\begin{displaymath}
E_n(x)=\frac{1}{(n+1)!}f^{(n+1)}(\xi)x^{(n+1)}
\end{displaymath}
\end{frame}

\begin{frame}
  \frametitle{Taylor's Theorem}
\begin{description}
\item[THEOREM 3] \textsf{Mean-Value Theorem}
\end{description}
If $f$ is in C[a,b] and if $f'$ exists on the open interval (a,b), then for $x$ and $c$ in the closed interval[a,b],
\begin{displaymath}
f(x)=f(c)+f'(\xi)(b-a)
\end{displaymath}
where $\xi$ is between $c$ and $x$.
Taking $x=b$ and $c=a$ and rearranging, we have the important equation
\begin{displaymath}
f(b)-f(a)=f'(\xi)(b-a) \qquad where\: a<\xi<b
\end{displaymath}
\end{frame}

\begin{frame}
  \frametitle{Taylor's Theorem}
\begin{description}
\item[THEOREM 4] \textsf{Rolle's Theorem}
\end{description}
If $f$ is continuous on [a,b], if $f'$ exists on  (a,b), and if $f(a)=f(b)$, then $f'(\xi)=0$ for some $\xi$ in the open interval(a,b).
\newline

This is an immediate consequence of an equation above. (Actually, in a formal development, Rolle's Theorem is proved first, and from it, Talor's Theorem is derived.) In both Rolle's Theorem and the Mean-value Theorem, there may be more than one point $\xi$ in the interval[a,b] that satisfies the given equations.
\end{frame}

\begin{frame}
  \frametitle{Taylor's Theorem}
\begin{description}
\item[THEOREM 5] \textsf{Taylor's Theorem with Integral Remainder}
\end{description}
If $f \in C^{(n+1)}$[a,b], then for any points $x$ and $c$ in the closed interval [a,b],
\begin{displaymath}
f(x)=\sum_{k=0}^n \frac{1}{k!}f^{(k)}(c)(x-c)^k +R_n(x)
\end{displaymath}
where 
\newcommand{\ud}{\mathrm{d}}
\begin{displaymath}
R_n(x)=\frac{1}{n!}\int_{c}^{x} f^{(n+1)}(t)(x-t)^n \ud t
\end{displaymath}
\end{frame}

\subsection{Other Forms of Taylor's Formula}
\begin{frame}
  \frametitle{Other Forms of Taylor's Formula}
\begin{description}
\item[THEOREM 6] \textsf{Alternative Form of Taylor's Theorem}
\end{description}
If $f \in C^{(n+1)}$[a,b], then for any points $x$ and $x+h$ in the closed interval [a,b],
\begin{displaymath}
f(x+h)=\sum_{k=0}^n \frac{h^k}{k!}f^{(k)}(x) +E_n(h)
\end{displaymath}
where 
\begin{displaymath}
E_n(h)=\frac{h^{n+1}}{(n+1)!} f^{(n+1)}(\xi)
\end{displaymath}
In which the point $\xi$ lies between $x$ and $x+h$. In detail,
\begin{displaymath}
f(x+h)=f(x)+hf'(x)+\frac{h^2}{2}f''(x)+\frac{h^3}{3!}f'''(x)+\ldots+\frac{h^n}{n!}f^{(n)}(x)+E_n(h)
\end{displaymath}
It is an important form for many applications.
\end{frame}

\begin{frame}
  \frametitle{Other Forms of Taylor's Formula}
\begin{description}
\item[THEOREM 7] \textsf{Taylor's Theorem in Two Variables}
\end{description}
Let $f \in C^{(n+1)}$([a,b]×[c,d]). If $(x,y)$ and $(x+h,y+k)$ are points in the rectangle [a,b]×[c,d]$\subseteq R^2$, then
\begin{displaymath}
f(x+h,y+k) = \sum_{i=0}^n \frac{1}{i!}\bigg(h\frac{\partial}{\partial x}+k\frac{\partial}{\partial y}\bigg)^i f(x+y)+E_n(h,k)
\end{displaymath}
where
\begin{displaymath}
E_n(h,k)=\frac{1}{(n+1)!} \bigg(h\frac{\partial}{\partial x}+k\frac{\partial}{\partial y}\bigg)^{(n+1)}f(x+\theta h,y+\theta k)
\end{displaymath}
in which $\theta$ lies between $0$ and $1$.
\end{frame}

\begin{frame}
  \frametitle{Other Forms of Taylor's Formula}
%\shoveleft
\begin{displaymath}
\bigg(h\frac{\partial}{\partial x}+k\frac{\partial}{\partial y}\bigg)^0f(x,y)=f(x,y)
\end{displaymath}
\begin{displaymath}
\bigg(h\frac{\partial}{\partial x}+k\frac{\partial}{\partial y}\bigg)^1f(x,y)=\bigg(h\frac{\partial}{\partial x}+k\frac{\partial}{\partial y}\bigg)(x,y)
\end{displaymath}
\begin{displaymath}
\bigg(h\frac{\partial}{\partial x}+k\frac{\partial}{\partial y}\bigg)^2f(x,y)=\bigg(h^2\frac{\partial^2f}{\partial x^2}+2hk\frac{\partial^2f}{\partial x\partial y}+k^2\frac{\partial^2f}{\partial y^2}\bigg)(x,y)
\end{displaymath}
and so on.Letting $f_x=\frac{ \partial f}{\partial x}$, $f_y=\frac{\partial f}{\partial y}$, $f_{xx}=\frac{ \partial^2f}{\partial x^2}$, $f_{xy}=\frac{ \partial^2f}{\partial x \partial y}$, $f_{yy}=\frac{ \partial^2f}{\partial y^2}$, we can write the first few terms of (5) as 
\begin{displaymath}
f(x+h,y+k) =f+(hf_x+kf_y)+\frac{1}{2}(h^2 f_{xx}+2hkf_{xy}+k^2f_{yy})+\ldots
\end{displaymath}
where on the right-hand side the function $f$ and each of the following partial derivatives are evaluated at (x,y).
\end{frame}
 
\end{document}
