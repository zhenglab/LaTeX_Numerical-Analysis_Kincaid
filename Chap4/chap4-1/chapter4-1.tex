\documentclass[notheorems,mathserif,table,compress]{beamer}  %dvipdfm选项是关键,否则编译统统通不过
%%------------------------常用宏包------------------------
%%注意, beamer 会默认使用下列宏包: amsthm, graphicx, hyperref, color, xcolor, 等等
\usepackage{fontspec,xunicode,xltxtra}  % for XeTeX
\usepackage{verbatim}
\usepackage{mathabx}
\usepackage{amsfonts,amssymb}
%%------------------------ThemeColorFont------------------------
%% Presentation Themes
% \usetheme[<options>]{<name list>}
\usetheme{Madrid}
%% Inner Themes双精度计算
% \useinnertheme[<options>]{<name>}
%% Outer Themes
% \useoutertheme[<options>]{<name>}
\useoutertheme{miniframes} 
%% Color Themes 
%\usecolortheme[<options>]{<name list>}
%% Font Themes
\usefonttheme{serif}
\setbeamertemplate{background canvas}[vertical shading][bottom=white,top=structure.fg!7] %%背景色, 上25%的蓝, 过渡到下白.
\setbeamertemplate{theorems}[numbered]
\setbeamertemplate{navigation symbols}{}   %% 去掉页面下方默认的导航条.
\usepackage{zhfontcfg}
%\setsansfont[Mapping=tex-text]{文泉驿正黑}  %% 需要fontspec宏包
     %如果装了Adobe Acrobat,可在font.conf中配置Adobe字体的路径以使用其中文字体
     %也可直接使用系统中的中文字体如SimSun,SimHei,微软雅黑 等
     %原来beamer用的字体是sans family;注意Mapping的大小写,不能写错
     %设置字体时也可以直接用字体名,以下三种方式等同:
     %\setromanfont[BoldFont={黑体}]{宋体}
     %\setromanfont[BoldFont={SimHei}]{SimSun}
     %\setromanfont[BoldFont={"[simhei.ttf]"}]{"[simsun.ttc]"}
%%------------------------MISC------------------------
\graphicspath{{figures/}}         %% 图片路径. 本文的图片都放在这个文件夹里了.
%%------------------------正文------------------------
\begin{document}
\XeTeXlinebreaklocale "zh"         % 表示用中文的断行
\XeTeXlinebreakskip = 0pt plus 1pt % 多一点调整的空间
%%----------------------------------------------------------
%% This is only inserted into the PDF information catalog. Can be left
%% out.
%%%
%% Delete this, if you do not want the table of contents to pop up at
%% the beginning of each subsection:
\AtBeginSection[]{                              % 在每个Section前都会加入的Frame
  \frame<handout:0>{
    \frametitle{Contents}\small
    \tableofcontents[current,currentsubsection]
  }
}

\AtBeginSubsection[]                            % 在每个子段落之前
{
  \frame<handout:0>                             % handout:0 表示只在手稿中出现
  {
    \frametitle{Contents}\small
    \tableofcontents[current,currentsubsection] % 显示在目录中加亮的当前章节
  }
}

%%----------------------------------------------------------
\title[Numerical Analysis]{Numerical Analysis}
\subtitle{Mathematics of Scientific Computing}
\author[qiu]{主讲人~~~~~\textcolor{olive}{邱欣欣}\\
    \quad 幻灯片制作~~\textcolor{olive}{邱欣欣}}
\institute[中国海洋大学]{\small\textcolor{violet}{中国海洋大学~~信息科学与工程学院}}
\date{2013~年~10~月~11~日}
%\titlegraphic{\vspace{-6em}\includegraphics[height=7cm]{ouc}\vspace{-6em}}
\frame{ \titlepage }
%%----------------------------------------------------------
\section*{Contents}
\frame{\frametitle{Solving Systems of Linear Equations}\tableofcontents}
%%----------------------------------------------------------
\section{Introduction}

%4
\begin{frame}
\frametitle{Introduction}
\begin{itemize}
\item Constructing a general-purpose algorithm for solving the problem $Ax=b$.
\item Analyzing the errors that are associated with the computer solution and study methods for controlling and reducing the error.
\item Introducing the important topic of iterative algorithm for this problem.
\end{itemize}
\end{frame}

%5
\begin{frame}
\frametitle{Introduction}
\begin{itemize}
\item The overall objective of this chapter is to discuss the numerical aspects of solving systems of linear equations having the form
\begin{equation*}
\left\{ \begin{aligned}
a_{11}x_1+a_{12}x_2+a_{13}x_3+\ldots+a_{1n}x_n&=b_1 \\
a_{21}x_1+a_{22}x_2+a_{23}x_3+\ldots+a_{2n}x_n&=b_2 \\
a_{31}x_1+a_{32}x_2+a_{33}x_3+\ldots+a_{3n}x_n&=b_3 \\
\vdots \\
a_{n1}x_1+a_{n2}x_2+a_{n3}x_3+\ldots+a_{nn}x_n&=b_n \\
\end{aligned} \right.
\end{equation*}

This is a system of $n$ equations in $n$ unknowns $x_1,x_2,\ldots,x_n$. The elements $a_{ij}$ and $b_i$ are assumed to be real numbers.
\end{itemize}
\end{frame}

%6
\begin{frame}
\frametitle{Introduction}
\begin{itemize}
\item Matrix are useful devices for representing systems of equations. The above system of linear equations can be written as
\begin{displaymath}
\begin{bmatrix}
a_{11} & a_{12} & a_{13} & \ldots & a_{1n}\\
a_{21} & a_{22} & a_{23} & \ldots & a_{2n}\\
a_{31} & a_{32} & a_{33} & \ldots & a_{3n}\\
\vdots & \vdots & \vdots & \ddots & \vdots \\
a_{n1} & a_{n2} & a_{n3} & \ldots & a_{nn}\\
\end{bmatrix}
\begin{bmatrix}
x_1\\
x_2\\
x_3\\
\vdots\\
x_n
\end{bmatrix}
=\begin{bmatrix}
b_1\\
b_2\\
b_3\\
\vdots\\
b_n
\end{bmatrix}
\end{displaymath}

We can denote these matrices by $A$, $x$, and $b$, so that the equation becomes simply $Ax=b$.
\end{itemize}
\end{frame}

\section{Matrix Algebra}

%8
\begin{frame}
\frametitle{Matrix Algebra}
\begin{itemize}
\item A matrix is a rectangular array of numbers such as
\begin{displaymath}
\left[\begin{array}{ccc}
3.0 & 1.1 & -0.12 \\
6.2 & 0.0 &  0.15\\
0.6 & -4.0 & 1.3\\
9.3 & 2.1 & 8.2 \\
\end{array} \right]
\left[\begin{array}{cccc}
3 & 6 & \cfrac{11}{7} & -17 
\end{array} \right]
\left[\begin{array}{ccc}
3.2\\
-4.7\\
0.11
\end{array} \right]
\end{displaymath}

These are, a $4\times3$ matrix, a $1\times4$ matrix and a $3\times1$ matrix. 
\end{itemize}

\begin{itemize}
\item A $1\times n$ matrix is called a \textsf{row vector}. A $m\times 1$ matrix is called a \textsf{column vector} or just a \textsf{vector}.
\end{itemize}
\begin{itemize}
\item If $A$ is a matrix, the notations $a_{ij}$, $(A)_{ij}$, or $A(i,j)$ is used to denote the element at the intersection of the $i$th row and $j$th column. 
\end{itemize}
\end{frame}

%9
\begin{frame}
\frametitle{Matrix Algebra}
\begin{itemize}
\item The \textsf{transpose} of a matrix is denoted by $A^{T}$ and is the matrix defined by $(A^{T})_{ij}=a_{ji}$. For example, if $A$ denotes 
\begin{displaymath}
\left[\begin{array}{ccc}
3.0 & 1.1 & -0.12 \\
6.2 & 0.0 &  0.15\\
0.6 & -4.0 & 1.3\\
9.3 & 2.1 & 8.2 \\
\end{array} \right]
\end{displaymath}

we have
\begin{displaymath}
A^T=
\left[\begin{array}{cccc}
3.0 & 6.2 & 0.6 & 9.3 \\
1.1 & 0.0 & -4.0 & 2.1\\
-0.12 & 0.15 & 1.3 & 8.2
\end{array} \right]
\end{displaymath}

If a matrix $A$ has the property $A^T=A$, we say that $A$ is \textsf{symmetric}.
\end{itemize}
\end{frame}

%10
\begin{frame}
\frametitle{Matrix Algebra}
\begin{itemize}
\item If $A$ is matrix and $\lambda$ is a scalar (that is, a real number in this context), then $\lambda A$ is defined by $(\lambda A)_{ij}=\lambda a_{ij}$.
\item If $A=(a_{ij})$ and $B=(b_{ij})$ are $m\times n$ matrices, then $A+B$ is defined by $(A+B)_{ij}=a_{ij}+b_{ij}$.
\item If $A$ is an $m\times p$ matrix and $B$ is a $p\times n$ matrix, then $AB$ is an $m\times n$ matrix defined by  
\begin{displaymath}
(AB)_{ij}=\sum^{p}_{k=1}a_{ik}b_{kj} \qquad (1\leq i\leq m, 1\leq j \leq n)
\end{displaymath}

\end{itemize}
\end{frame}

%11
\begin{frame}
\frametitle{Matrix Algebra}
\begin{itemize}
\item Here are some examples of the algebraic operations:
\begin{displaymath}
\begin{bmatrix}
1 & 3\\
2 & -1\\
4 & -4
\end{bmatrix}
+
\begin{bmatrix}
6 & 0\\
3 & -7\\
8 & 2
\end{bmatrix}
=
\begin{bmatrix}
7 & 3\\
5 & -8\\
12 & -2
\end{bmatrix}
\end{displaymath}

\begin{displaymath}
3
\begin{bmatrix}
1 & 3\\
2 & -1\\
4 & -4
\end{bmatrix}
=
\begin{bmatrix}
3 & 9\\
6 & -3\\
12 & -12
\end{bmatrix}
\end{displaymath}

\begin{displaymath}
\begin{bmatrix}
2 & 1 & 3\\
1 & 5 & -6\\
2 & 1 & 5 \\
0 & 1 & -2
\end{bmatrix}
\begin{bmatrix}
1 & 0\\
-5 & 4\\
0 & 3
\end{bmatrix}
=
\begin{bmatrix}
-3 & 13\\
-24 & 2\\
-3 & 19 \\
-5 & -2
\end{bmatrix}
\end{displaymath}

\end{itemize}
\end{frame}

%12
\begin{frame}
\frametitle{Matrix Algebra}
\begin{itemize}
\item In the systems of linear equations, a concept of equivalence is important.
\item Let us consider two systems, each of them consisting of $n$ equation with $n$ unknowns expressed by
\begin{displaymath}
Ax=b \qquad Bx=d
\end{displaymath}

If the two systems have precisely the same solutions of equations, we call them \textsf{equivalent systems}.
\item To solve a system of equations, we can instead solve any equivalent system; no solutions are lost and no new ones appear.
\end{itemize}
\end{frame}

%13
\begin{frame}
\frametitle{Matrix Algebra} 
These \textsf{elementary operations} are of the following three types. (Here, $\varepsilon_i$ denotes the $i$th equation in the system.)
\begin{enumerate}
\item Interchanging two equations in the system: $\varepsilon_i\leftrightarrow\varepsilon_j$
\item Multiplying an equation by a nonzero number: $\lambda\varepsilon_i\rightarrow\varepsilon_i$
\item Adding to an equation a multiple of some other equation: $\varepsilon_i+\lambda\varepsilon_j\rightarrow\varepsilon_i$
\end{enumerate}

\end{frame}

%14
\begin{frame}
\frametitle{Theorem on Equivalent Systems}
\theoremstyle{plain}
\newtheorem{theorem}{THEOREM}
\begin{theorem}[Theorem on Equivalent Systems]
If one system of equations is obtained from another by a finite sequence of elementary operations, then the two systems are equivalent.
\end{theorem}
\begin{proof}
Suppose that an elementary operation transforms the system $Ax=b$ into the system $Bx=d$.
\begin{itemize}
\item For the Type 1, then the two systems consist of the same equations. If $x$ solves the first system, then it solves the second, and vice versa.
\end{itemize}

\end{proof}
\end{frame}

%15
\begin{frame}
\frametitle{Theorem on Equivalent Systems}
\begin{proof}
For the Type 2, then suppose that the $i$th equation has been multiplied by a scalar $\lambda$, with $\lambda \neq 0$. The $i$th and $j$th equations in $Ax=b$ are
\begin{eqnarray}
a_{i1}x_1+\ldots+a_{in}x_n&=&b_i\\
a_{j1}x_1+\ldots+a_{jn}x_n&=&b_j
\end{eqnarray}

and the $i$th equation in $Bx=d$ is
\begin{equation}
\lambda a_{i1}x_1+\ldots+\lambda a_{in}x_n=\lambda b_i
\end{equation}

Any vector $x$ that satisfies Equation (1) satisfies Equation (3), and vice versa, because $\lambda \neq 0$. 
\end{proof}
\end{frame}

%16
\begin{frame}
\frametitle{Theorem on Equivalent Systems}
\begin{proof}
The $i$th and $j$th equations in $Ax=b$ are
\begin{eqnarray}
a_{i1}x_1+\ldots+a_{in}x_n&=&b_i\\
a_{j1}x_1+\ldots+a_{jn}x_n&=&b_j
\end{eqnarray}

For the Type 3, assume that $\lambda$ times the $j$th equation has been added to the $i$th. Then the $i$th equation in $Bx=d$ is
\begin{eqnarray}
(a_{i1}+\lambda a_{j1})x_1+\ldots+(a_{in}+\lambda a_{jn})x_n=b_i+\lambda b_j
\end{eqnarray}

If $Ax=b$, then Equation (4) and (5) are true. Hence, (6) is true.\\
If we suppose that $x$ solves $Bx=d$, then Equation (6) and (5) are true. If $\lambda$ times Equation (5) is subtracted from Equation (6), the result is Equation (4). Hence, $Ax=b$.
\end{proof}
\end{frame}

%17
\begin{frame}
\frametitle{Matrix Properties}
\begin{itemize}
\item The $n\times n$ matrix 
\begin{displaymath}
I=
\left[\begin{array}{ccccc}
1 & 0 & 0 & \ldots & 0 \\
0 & 1 & 0 & \ldots & 0 \\
0 & 0 & 1 & \ldots & 0 \\
\vdots & \vdots & \vdots & \ddots & \vdots \\
0 & 0 & 0 & \ldots & 1 
\end{array} \right]
\end{displaymath}

is called an \textsf{identity matrix}. It has the property that $IA=A=AI$ for any matrix $A$ of size $n\times n$. 
\end{itemize}
\end{frame}

%18
\begin{frame}
\frametitle{Matrix Properties}
\begin{itemize}
\item If $A$ and $B$ are two matrices such that $AB=I$, then we say that $B$ is a \textsf{right inverse} of $A$ and $A$ is a \textsf{left inverse} of $B$. For example,

\begin{displaymath}
\begin{bmatrix}
1 & 0 & 0\\
0 & 1 & 0
\end{bmatrix} 
\begin{bmatrix}
1 & 0 \\
0 & 1\\
\alpha & \beta
\end{bmatrix}
=
\begin{bmatrix}
1 & 0 \\
0 & 1 
\end{bmatrix} 
\end{displaymath}

We see that if a matrix has a right inverse, then the latter is not necessarily unique.
\end{itemize}
\end{frame}

%19
\begin{frame}
\frametitle{Matrix Properties}
\begin{theorem}[Theorem on Right Inverse]
A square matrix can possess at most one right inverse.
\end{theorem}
\begin{proof}
Let $AB=I$, where $A$, $B$, and $I$ are all $n\times n$ matrices. Denote by $A^{(j)}$ the $j$th column of $A$ and by $I^{(k)}$ the $k$th column of $I$. The equation $AB=I$ means that 
\begin{equation}
\sum_{j=1}^{n}b_{jk}A^{(j)}=I^{(k)} \qquad (1\leq k\leq n)
\end{equation}

Each column of $I$ is a linear combination of the columns of $A$. Since the columns of $I$ span $\mathbb{R}^n$, the same is true of the columns of $A$. Hence, the columns of $A$ form a basis for $\mathbb{R}^n$, consequently, the coefficients $b_{jk}$ in Equation (7) are uniquely determined. 
\end{proof}
\end{frame}

%20
\begin{frame}
\frametitle{Matrix Properties}
\begin{displaymath}
\begin{bmatrix}
a_{11} & a_{12} & \ldots & a_{1n}\\
a_{21} & a_{22} & \ldots & a_{2n}\\
\vdots & \vdots & \ddots & \vdots \\
a_{n1} & a_{n2} & \ldots & a_{nn}
\end{bmatrix}
\begin{bmatrix}
b_{11} & b_{12} & \ldots & b_{1n}\\
b_{21} & b_{22} & \ldots & b_{2n}\\
\vdots & \vdots & \ddots & \vdots \\
b_{n1} & b_{n2} & \ldots & b_{nn}
\end{bmatrix}
=\begin{bmatrix}
1 & 0 & \ldots & 0 \\
0 & 1 & \ldots & 0 \\
\vdots & \vdots & \ddots & \vdots \\
0 & 0 & \ldots & 1 
\end{bmatrix}
\end{displaymath}

\begin{displaymath}
b_{11}\begin{bmatrix}
a_{11}\\
a_{21}\\
\vdots\\
a_{n1}
\end{bmatrix}
+b_{21}\begin{bmatrix}
a_{12}\\
a_{22}\\
\vdots\\
a_{n2}
\end{bmatrix}+\ldots+b_{n1}
\begin{bmatrix}
a_{1n}\\
a_{2n}\\
\vdots\\
a_{nn}
\end{bmatrix}=
\begin{bmatrix}
1\\
0\\
\vdots\\
0
\end{bmatrix}
\end{displaymath}

\begin{displaymath}
\sum_{j=1}^{n}b_{j1}A^{(j)}=I^{(1)} \qquad (1\leq k\leq n)
\end{displaymath}

\end{frame}

%21
\begin{frame}
\frametitle{Matrix Properties}
\begin{theorem}[Theorem on Matrix Inverse]
If $A$ and $B$ are square matrices such that $AB=I$, then $BA=I$.
\end{theorem}
\begin{proof}
Let $C=BA-I+B$. Then
\begin{equation*}
AC=ABA-AI+AB=A-A+I=I
\end{equation*}

Thus, $C$ (as well as $B$) is a right inverse of $A$. By Theorem 2, $B=C$; hence, $BA=I$.  
\end{proof}
\end{frame}

%22
\begin{frame}
\frametitle{Matrix Properties}
\begin{itemize}
\item If a square matrix $A$ has a right inverse $B$, then $B$ is unique and $BA=AB=I$. We call $B$ the \textsf{inverse} of $A$ and say that $A$ is \textsf{invertible} or \textsf{nonsingular}. We write $B=A^{-1}$ and $A=B^{-1}$.
\item If $A$ is invertible, the system of equations $Ax=b$ has the solution $x = A^{-1}b$. 
\end{itemize}
\end{frame}

%23
\begin{frame}
\frametitle{Matrix Properties}
An \textsf{elementary matrix} is defined to be an $n\times n$ matrix that arises when an elementary operation is applied to the $n\times n$ identity matrix.
\newline

The elementary operations expressed in terms of the rows of a matrix $A$, are
\begin{enumerate}
\item The interchange of two rows in $A$: $A_s\leftrightarrow A_t$
\item Multiplying one row by a nonzero constant: $\lambda A_s \rightarrow A_s$
\item Adding to one row a multiple of another: $A_s + \lambda A_t \rightarrow A_s$
\end{enumerate}

\end{frame}

%24
\begin{frame}
\frametitle{Matrix Properties}
\begin{itemize}
\item Each elementary row operation on $A$ can be accomplished by multiplying $A$ on the left by an elementary matrix.
\end{itemize}
\begin{eqnarray*}
\begin{bmatrix}
1 & 0 & 0\\
0 & 0 & 1\\
0 & 1 & 0
\end{bmatrix} 
\begin{bmatrix}
a_{11} & a_{12} & a_{13}\\
a_{21} & a_{22} & a_{23}\\
a_{31} & a_{32} & a_{33}
\end{bmatrix}
&=&
\begin{bmatrix}
a_{11} & a_{12} & a_{13}\\
a_{31} & a_{32} & a_{33}\\
a_{21} & a_{22} & a_{23}
\end{bmatrix} \\
\begin{bmatrix}
1 & 0 & 0\\
0 & \lambda & 0\\
0 & 0 & 1
\end{bmatrix} 
\begin{bmatrix}
a_{11} & a_{12} & a_{13}\\
a_{21} & a_{22} & a_{23}\\
a_{31} & a_{32} & a_{33}
\end{bmatrix}
&=&
\begin{bmatrix}
a_{11} & a_{12} & a_{13}\\
\lambda a_{21} & \lambda a_{22} & \lambda a_{23}\\
a_{31} & a_{32} & a_{33}
\end{bmatrix} \\
\begin{bmatrix}
1 & 0 & 0\\
0 & 1 & 0\\
0 & \lambda & 1
\end{bmatrix} 
\begin{bmatrix}
a_{11} & a_{12} & a_{13}\\
a_{21} & a_{22} & a_{23}\\
a_{31} & a_{32} & a_{33}
\end{bmatrix}
&=&
\begin{bmatrix}
a_{11} & a_{12} & a_{13}\\
a_{21} & a_{22} & a_{23}\\
\lambda a_{21}+a_{31} & \lambda a_{22}+a_{32} & \lambda a_{23}+a_{33}
\end{bmatrix} 
\end{eqnarray*}

\end{frame}

%25
\begin{frame}
\frametitle{Matrix Properties}
\begin{itemize}
\item Introducing elementary matrices $E_1$, $E_2$, $\ldots$, $E_m$. If a matrix is invertible, a sequence of elementary row operations can be applied to $A$, reducing it to $I$. Thus, we have 
\begin{displaymath}
E_mE_{m-1}\ldots E_2E_1A=I
\end{displaymath}

From this it follows that $A^{-1}=E_mE_{m-1}\ldots E_2E_1$. 
\item Consequently, $A^{-1}$ can be obtained by subjecting $I$ to the same sequence of elementary row operations.
\end{itemize}
\end{frame}

%26
\begin{frame}
\frametitle{Matrix Properties}
\begin{eqnarray*}
A&=&
\begin{bmatrix}
1 & 2 & 3\\
1 & 3 & 3\\
2 & 4 & 7
\end{bmatrix} \qquad
\begin{bmatrix}
1 & 0 & 0\\
0 & 1 & 0\\
0 & 0 & 1
\end{bmatrix} 
=I\\
E_1&=&
\begin{bmatrix}
1 & 0 & 0\\
-1 & 1 & 0\\
0 & 0 & 1
\end{bmatrix} \quad \;
\begin{bmatrix}
1 & 0 & 0\\
0 & 1 & 0\\
-2 & 0 & 1
\end{bmatrix}
=E_2\\
E_1A&=&
\begin{bmatrix}
1 & 2 & 3\\
0 & 1 & 0\\
2 & 4 & 7
\end{bmatrix} \qquad
\begin{bmatrix}
1 & 0 & 0\\
-1 & 1 & 0\\
0 & 0 & 1
\end{bmatrix} 
=E_1I\\
E_2E_1A&=&
\begin{bmatrix}
1 & 2 & 3\\
0 & 1 & 0\\
0 & 0 & 1
\end{bmatrix} \qquad
\begin{bmatrix}
1 & 0 & 0\\
-1 & 1 & 0\\
-2 & 0 & 1
\end{bmatrix} 
=E_2E_1I
\end{eqnarray*}

\end{frame}

%27
\begin{frame}
\frametitle{Matrix Properties}
\begin{eqnarray*}
E_3&=&
\begin{bmatrix}
1 & -2 & 0\\
0 & 1 & 0\\
0 & 0 & 1
\end{bmatrix} \quad \,
\begin{bmatrix}
1 & 0 & -3\\
0 & 1 & 0\\
0 & 0 & 1
\end{bmatrix}
=E_4\\
E_3E_2E_1A&=&
\begin{bmatrix}
1 & 0 & 3\\
0 & 1 & 0\\
0 & 0 & 1
\end{bmatrix} \qquad
\begin{bmatrix}
3 & -2 & 0\\
-1 & 1 & 0\\
-2 & 0 & 1
\end{bmatrix} 
=E_3E_2E_1I\\
E_4E_3E_2E_1A&=&
\begin{bmatrix}
1 & 0 & 0\\
0 & 1 & 0\\
0 & 0 & 1
\end{bmatrix} \qquad
\begin{bmatrix}
9 & -2 & -3\\
-1 & 1 & 0\\
-2 & 0 & 1
\end{bmatrix} 
=E_4E_3E_2E_1I=A^{-1}
\end{eqnarray*}

\end{frame}

%28
\begin{frame}
\frametitle{Matrix Properties}
\begin{theorem}[Theorem on Nonsingular Matrix Properties]
For an $n\times n$ matrix $A$, the following properties are equivalent:
\begin{enumerate}
\item The inverse of $A$ exists; that is, $A$ is nonsingular.
\item The determinant of $A$ is nonzero.
\item The rows of $A$ form a basis for $\mathbb{R}^n$.
\item The columns of $A$ form a basis for $\mathbb{R}^n$.
\item As a map from $\mathbb{R}^n$ to $\mathbb{R}^n$, $A$ is injective (one to one).
\item As a map from $\mathbb{R}^n$ to $\mathbb{R}^n$, $A$ is subjective (onto).
\item The equation $Ax=0$ implies $x=0$.
\item For each $b\in \mathbb{R}^n$, there is exactly one $x\in \mathbb{R}^n$ such that $Ax=b$.
\item $A$ is a product of elementary matrices.
\item 0 is not an eigenvalue of $A$.
\end{enumerate}
\end{theorem}

\end{frame}

%29
\begin{frame}
\frametitle{Matrix Properties}
\begin{itemize}
\item An important fundamental concept is the \textsf{positive definiteness} of a matrix. A matrix $A$ is \textsf{positive definite} if $x^TAx>0$ for every nonzero vector $x$. For example, the matrix
\begin{displaymath}
A=\begin{bmatrix}
2 & 1\\
1 & 2 
\end{bmatrix}
\end{displaymath}

is positive definite since
\begin{displaymath}
x^TAx=
\begin{bmatrix}
x_1 & x_2\\
\end{bmatrix}
\begin{bmatrix}
2 & 1\\
1 & 2 
\end{bmatrix}
\begin{bmatrix}
x_1\\
x_2 
\end{bmatrix}
=(x_1+x_2)^2+x_1^2+x_2^2>0
\end{displaymath}

for all $x_1$ and $x_2$ except $x_1=x_2=0$. Here, $x^TAx$ is called a \textsf{quadratic form}.
\end{itemize}
\end{frame}

%30
\begin{frame}
\frametitle{Partitioned matrices}
\begin{itemize}
\item It is convenient to partition matrices into submatrices and compute products as if the submatrices were numbers.
\begin{displaymath}
\begin{bmatrix}
\begin{bmatrix}
1 & 2
\end{bmatrix} &
\begin{bmatrix}
1 & -1 & 0 & 1
\end{bmatrix}\\
\begin{bmatrix}
-1 & 1\\
0 & 1 \\
1 & -1\\
1 & 0
\end{bmatrix}&
\begin{bmatrix}
1 & 0 & -1 & 1\\
-1 & 1 & 0 & 1\\
0 & 0 & 1 & 0\\
1 & 2 & 1 & 0
\end{bmatrix}
\end{bmatrix}
\begin{bmatrix}
\begin{bmatrix}
1 & 0 & 1\\
-1 & 1 & 2
\end{bmatrix} &
\begin{bmatrix}
2 & 1\\
0 & 1
\end{bmatrix}\\
\begin{bmatrix}
1 & 0 & 1\\
-1 & 1 & 0\\
2 & 1 & 0\\
0 & 1 & 1
\end{bmatrix}&
\begin{bmatrix}
1 & 2\\
0 & 1 \\
-2 & 1\\
-1 & 1
\end{bmatrix}
\end{bmatrix}
\end{displaymath}

\begin{displaymath}
=
\begin{bmatrix}
\begin{bmatrix}
1 & 2 & 7
\end{bmatrix} &
\begin{bmatrix}
2 & 5
\end{bmatrix}\\
\begin{bmatrix}
-3 & 1 & 3\\
-3 & 3 & 2\\
4 & 0 & -1\\
2 & 3 & 2
\end{bmatrix}&
\begin{bmatrix}
0 & 2\\
-2 & 1 \\
0 & 1\\
1 & 6
\end{bmatrix}
\end{bmatrix}
\end{displaymath}

\end{itemize}
\end{frame}

%31
\begin{frame}
\frametitle{Matrix Properties}
\begin{itemize}
\item If the submatrices are denoted by single letters, we have a product of the form
\begin{displaymath}
\begin{bmatrix}
A_{11} & A_{12}\\
A_{21} & A_{22} 
\end{bmatrix}
\begin{bmatrix}
B_{11} & B_{12}\\
B_{21} & B_{22} 
\end{bmatrix}
=
\begin{bmatrix}
C_{11} & C_{12}\\
C_{21} & C_{22} 
\end{bmatrix}
\end{displaymath}

One can verify that $C_{ij}=\sum_{s=1}^{2}A_{is}B_{sj}$. \\
\item For example, $C_{11}=A_{11}B_{11}+A_{12}B_{21}$.
\end{itemize}
\end{frame}

%32
\begin{frame}
\frametitle{Matrix Properties}
\begin{theorem}[Theorem on Multiplication of Partitioned Matrices]
If each product $A_{is}B_{sj}$ can be formed, and if $C_{ij}=\sum_{s=1}^nA_{is}B_{sj}$,then $C=AB$.       
\end{theorem}
\begin{proof}
Let the dimensions of $A_{ij}$ be $m_i\times n_j$. Let the dimensions of $B_{ij}$ be $\widehat m_i\times \widehat n_j$. Since $A_{is}B_{sj}$ exists, we must have $n_s=\widehat m_s$ for all $s$. Then $C_{ij}$ will have dimension $m_i\times \widehat n_j$. Now select an arbitrary element $c_{ij}$ in the matrix $C$. Suppose that $c_{ij}$ lie in the block $C_{rs}$ and is in the $p$th row and $q$th column of $C_{rs}$. Then we must have 
\begin{eqnarray*}
i&=&m_1+m_2+\ldots+m_{r-1}+p\\
j&=&\widehat n_1+\widehat n_2+\ldots+\widehat n_{s-1}+q
\end{eqnarray*}

\end{proof}
\end{frame}

%33
\begin{frame}
\frametitle{Matrix Properties}
\begin{proof}
Then we have  
\begin{displaymath}
c_{ij}=(C_{rs})_{pq}=(\sum_{t=1}^{n}A_{rt}B_{ts})_{pq}=\sum_{t=1}^{n}(A_{rs}B_{ts})_{pq}=\sum_{t=1}^{n}\sum_{\alpha=1}^{n_t}(A_{rt})_{p\alpha}(B_{ts})_{\alpha q}
\end{displaymath}

The elements $(A_{rt})_{p\alpha}$ lie in row $i$ of $A$. These elements fill out the entire row $i$ of $A$ since $1\leq t \leq n$ and $1\leq \alpha \leq n_t$. The elements $(B_{ts})_{\alpha q}$ lie in column $j$ of $B$. Also, the entire column $j$ of $B$ is present and appears in its natural order. Hence,
\begin{displaymath}
c_{ij}=\sum_{\beta=1}^{n}(A)_{i\beta}(B)_{\beta j}=(AB)_{ij}
\end{displaymath}

\end{proof}
\end{frame}
\end{document}
